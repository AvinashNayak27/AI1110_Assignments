%%%%%%%%%%%%%%%%%%%%%%%%%%%%%%%%%%%%%%%%%%%%%%%%%%%%%%%%%%%%%%%
%
% Welcome to Overleaf --- just edit your LaTeX on the left,
% and we'll compile it for you on the right. If you open the
% 'Share' menu, you can invite other users to edit at the same
% time. See www.overleaf.com/learn for more info. Enjoy!
%
%%%%%%%%%%%%%%%%%%%%%%%%%%%%%%%%%%%%%%%%%%%%%%%%%%%%%%%%%%%%%%%


% Inbuilt themes in beamer
\documentclass{beamer}
\newcommand{\mydet}[1]{\ensuremath{\begin{vmatrix}#1\end{vmatrix}}}
\usepackage[super]{nth}
\usepackage{amsmath}
\usepackage{amssymb}
% Theme choice:
\usetheme{CambridgeUS}

% Title page details: 
\title{Assignment 5}% \\ Indian Institute of Technology Hyderabad} 
\author{Malothu Avinash \\ AI21BTECH11018}
\date{\today}
% \logo{\large \LaTeX{}}


\begin{document}

% Title page frame
\begin{frame}
    \titlepage 
\end{frame}

% Remove logo from the next slides
% \logo{}


% Outline frame
\begin{frame}{Outline}
    \tableofcontents
\end{frame}


% Lists frame
\section{Question}
\begin{frame}{Question}
A voltage source V is measured six times.the measurements are modeled by the random variable x=V+$\nu$.Assume that the error $\nu$ is is N(0,$\sigma$).Find the 0.95 interval estimate of $\sigma^2$
\end{frame}

\section{Answer}
\begin{frame}{Answer}
(a)If the source is known standard with V=110V.\\
As the point estimator of $\nu$ the average is\\
\begin{align}
    &\hat{\nu}=\frac{1}{n}\sum_{i=1}^{n} (x_i-\eta)^{2}\\
    &Inserting\;the\;measured\;values\;{x_i=110+\nu_i}\;in\;above\; formula\\
    &we\;get\;\hat{\nu}=0.25\\
    &From\;the\;table\;of\;Chi-square\;percentiles\;\chi^2_\mu(n)\\
    &We\;get\; \chi^2_{0.025}(6)=1.24\;and\;\chi^2_{0.975}(6)=14.45
\end{align}
\end{frame}
\begin{frame}
\begin{align}
      &From\;the\;interval\;\frac{n\hat{\nu}}{\chi_{1-\delta/2}(n)}<\sigma^2<\frac{n\hat{\nu}}{\chi_{\delta/2}(n)}\\
      &We\;get\;{0.104}<\sigma^2<{1.2}\\
      &\implies Corresponding\;interval\;{0.332}<\sigma<{1.096}V\\
\end{align}
\end{frame}
\begin{frame}
(a)If the source is unknown standard .We compute from\\
\begin{align}
    &s^2=\frac{1}{n-1}\sum_{i=1}^{n} (x_i-\Bar{x})^{2}\\
    &Inserting\;the\;measured\;values\;{x_i=110+\nu_i}\;in\;above\; formula\\
    &we\;get\;s^2=0.30\\
    &From\;the\;table\;of\;Chi-square\;percentiles\;\chi^2_\mu(n)\\
    &We\;get\; \chi^2_{0.025}(5)=0.83\;and\;\chi^2_{0.975}(5)=12.83
\end{align}
\end{frame}
\begin{frame}
\begin{align}
      &From\;the\;interval\;\frac{(n-1)s^2}{\chi^2_{1-\delta/2}(n-1)}<\sigma^2<\frac{(n-1)s^2}{\chi^2_{\delta/2}(n-1)}\\
      &We\;get\;{0.117}<\sigma^2<{1.8}\\
      &\implies Corresponding\;interval\;{0.342}<\sigma<{1.344}V\\
\end{align}
\end{frame}
\end{document}
