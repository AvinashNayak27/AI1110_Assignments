%%%%%%%%%%%%%%%%%%%%%%%%%%%%%%%%%%%%%%%%%%%%%%%%%%%%%%%%%%%%%%%
%
% Welcome to Overleaf --- just edit your LaTeX on the left,
% and we'll compile it for you on the right. If you open the
% 'Share' menu, you can invite other users to edit at the same
% time. See www.overleaf.com/learn for more info. Enjoy!
%
%%%%%%%%%%%%%%%%%%%%%%%%%%%%%%%%%%%%%%%%%%%%%%%%%%%%%%%%%%%%%%%


% Inbuilt themes in beamer
\documentclass{beamer}
\newcommand{\mydet}[1]{\ensuremath{\begin{vmatrix}#1\end{vmatrix}}}
\usepackage[super]{nth}

% Theme choice:
\usetheme{CambridgeUS}

% Title page details: 
\title{Assignment 2}% \\ Indian Institute of Technology Hyderabad} 
\author{Malothu Avinash \\ AI21BTECH11018}
\date{\today}
% \logo{\large \LaTeX{}}


\begin{document}

% Title page frame
\begin{frame}
    \titlepage 
\end{frame}

% Remove logo from the next slides
% \logo{}


% Outline frame
\begin{frame}{Outline}
    \tableofcontents
\end{frame}


% Lists frame
\section{Question}
\begin{frame}{Question}
    Using properties of determinants prove that:
    \begin{align}
    \mydet{x & x(x^2+1) & x+1\\y & y(y^2+1) & y+1\\z & z(z^2+1) & z+1} = (x-y)(y-z)(z-x)(x+y+z)
    \end{align}
\end{frame}

\section{Answer}
\begin{frame}{Answer}
    \begin{align}
    M = \mydet{x & x(x^2+1) & x+1\\y & y(y^2+1) & y+1\\z & z(z^2+1) & z+1}
    \end{align}
Using Split property of determinant at column 3 we get
    \begin{align}
        M = \mydet{x & x(x^2+1) & x\\y & y(y^2+1) & y\\z & z(z^2+1) & z}+ \mydet{x & x(x^2+1) & 1\\y & y(y^2+1) & 1\\z & z(z^2+1) & 1}
    \end{align}
As \nth{1} and \nth{3} coloumns of \nth{1} determinant are same it's value becomes zero then
\end{frame}
\begin{frame}

    \begin{align}
    M = \mydet{x & x^3+x & 1\\y & y^3+y & 1\\z & z^3+z & 1}
    \end{align}
Using Split property of determinant at column 2 we get 
    \begin{align}
    M = \mydet{x & x^3 & 1\\y & y^3 & 1\\z & z^3 & 1}+\mydet{x & x & 1\\y & y & 1\\z & z & 1}
    \end{align}
    Similarly as \nth{1} and \nth{2} coloumns of \nth{2} determinant are same it's value becomes zero then
\end{frame}
\begin{frame}
\begin{align}
     M = \mydet{x & x^3 & 1\\y & y^3 & 1\\z & z^3 & 1}
\end{align}
     Using row transformation properties i.e changing row1 to (row1-row2) and row2 to (row2-row3) we get
     \begin{align}
    M = \mydet{x-y & x^3-y^3 & 0\\y-z & y^3-z^3 & 0\\z & z^3 & 1}
     \end{align}
     \end{frame}
     \begin{frame}
              Evaluating the determinant of matrix at (3,3) position we get value of determinant as 
     \begin{align}
        & = (y^3-z^3)(x-y)-(y-z)(x^3-y^3)\\
        & =(y-z)(y^2+z^2+y.z)(x-y)-(y-z)(x^2+y^2+x.y)\\
        & =(y-z)(x-y)[y^2+z^2+y.z-x^2-y^2-x.y]\\
        & =(y-z)(x-y)[(z-x)(z+x)+y(z-x)]\\
        & =(y-z)(x-y)(z-x)[z+x+y]\\
        & =(x-y)(y-z)(z-x)(x+y+z)\\
        & =R.H.S
     \end{align}
     Hence proved!
     \end{frame}
\begin{frame}{Code Output:}
The following is a result of c code which takes inputs for x,y,z and checks whether both LHS and RHS are equal
\includegraphics[width=8cm]{solution.png}
\end{frame}

\end{document}
