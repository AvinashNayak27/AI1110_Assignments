\documentclass[12pt, twocolumn]{article}
\usepackage[utf8]{inputenc}
\usepackage{amsmath}
\usepackage[margin = 1.1in]{geometry}
\usepackage{float}
\usepackage{graphicx}
\graphicspath{{images/}}
\setlength{\columnsep}{0.75cm}
\usepackage{hyperref}
\newcommand{\solution}{\noindent \textbf{Solution: }}
\usepackage{tfrupee}
\hypersetup{
colorlinks=true,
    linkcolor= red,
    }

\title{AI1110 Assignment 1}
\author{MALOTHU AVINASH AI21BTECH11018}
\date{March 2022}

\begin{document}
\maketitle
\begin{problem}
\textbf{Q5(B):} Rekha opened a recurring deposit account for 20 months. The rate of interest is 9\% per annum and Rekha receives \rupee441 as interest at the time of maturity.Find the amount Rekha deposited each month.
\end{problem}

\solution 
According to given question

Rekha opened a recurring deposit account for 20 months(n),

Rate of interest(r) is 9\% per annum,and Rekha receives \rupee 441 as interest at the time of maturity.

From simple interest formula I=p\cdot\ t\cdot\frac{r}{100}

\implies Total Interest(i) = x\cdot\frac{1}{12}\cdot\frac{9}{100} + x\cdot\frac{2}{12}\cdot\frac{9}{100} + x\cdot\frac{3}{12}\cdot\frac{9}{100} +
---- +x\cdot\frac{20}{12}\cdot\frac{9}{100}

Given Total Interest is 441 then

\implies 441 = x\cdot\frac{9}{1200}(1+2+3+----+20)

\implies x = \frac{441\cdot1200\cdot2}{9\cdot20\cdot21}

Finally we get x = 280

c code output as follows
\includegraphics[width=\textwidth]{solution.png}
\end{document}
