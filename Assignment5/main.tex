%%%%%%%%%%%%%%%%%%%%%%%%%%%%%%%%%%%%%%%%%%%%%%%%%%%%%%%%%%%%%%%
%
% Welcome to Overleaf --- just edit your LaTeX on the left,
% and we'll compile it for you on the right. If you open the
% 'Share' menu, you can invite other users to edit at the same
% time. See www.overleaf.com/learn for more info. Enjoy!
%
%%%%%%%%%%%%%%%%%%%%%%%%%%%%%%%%%%%%%%%%%%%%%%%%%%%%%%%%%%%%%%%


% Inbuilt themes in beamer
\documentclass{beamer}
\newcommand{\mydet}[1]{\ensuremath{\begin{vmatrix}#1\end{vmatrix}}}
\usepackage[super]{nth}
\usepackage{amsmath}
\usepackage{amssymb}
% Theme choice:
\usetheme{CambridgeUS}

% Title page details: 
\title{Assignment 5}% \\ Indian Institute of Technology Hyderabad} 
\author{Malothu Avinash \\ AI21BTECH11018}
\date{\today}
% \logo{\large \LaTeX{}}


\begin{document}

% Title page frame
\begin{frame}
    \titlepage 
\end{frame}

% Remove logo from the next slides
% \logo{}


% Outline frame
\begin{frame}{Outline}
    \tableofcontents
\end{frame}


% Lists frame
\section{Question}
\begin{frame}{Question}
Using price's theorem,Derive E(x^2\cdot y^2) = E(x^2)\cdot E(y^2)+2E(xy)
\end{frame}

\section{Answer}
\begin{frame}{Answer}
From price's theorem\\

  \begin{align}
      &\frac{\partial^n I(\mu)}{\partial \mu^n}=E(\frac{\partial^2^n g(x,y)}{\partial^n x\cdot \partial^n y})\\
      &Let\;n=1\;and\;g(x,y)=x^2y^2\\
      &\frac{\partial I(\mu)}{\partial \mu}=E(\frac{\partial^2 (x^2y^2)}{\partial x\cdot \partial y})\\
      &\frac{\partial I(\mu)}{\partial \mu}=4E(xy)\\
      &\frac{\partial I(\mu)}{\partial \mu}=4\mu\\
      &I(\mu)=2\mu^2+ I(0)
  \end{align}
\end{frame}
\begin{frame}
\begin{align}
      &As\;I(\mu)=E(x^2\cdot y^2)\\
      &at\;\mu=0\; random\;variables\;x\;and\;y\;are\;independent\\
      &\implies I(0)=E(x^2)\cdot E(y^2)\\
      & From I(\mu)=2\mu^2+ I(0)\\
      &E(x^2\cdot y^2) = 2[E(x\cdot y)]^2+E(x^2)\cdot E(y^2)\\
      &\therefore E(x^2\cdot y^2) = E(x^2)\cdot E(y^2)+2[E(x\cdot y)]^2\\
\end{align}
\end{frame}

\end{document}
